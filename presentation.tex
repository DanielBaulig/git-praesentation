\begin{document}

\frame{
    \titlepage
}

\frame{
    \frametitle{Übersicht}
    \tableofcontents
}

\section{Einführung}
\subsection{Über mich}

\begin{frame}{Hello, my name is...}
    \begin{itemize}
        \item<1-> Daniel Baulig
        \item<2-> B.Sc. Informatik (TU Darmstadt, FH Frankfurt)
        \item<3-> IT-Security, Web (JS, HTML5, node.js), AI, FLOSS
        \item<4-> \temporal<4-5>{}{wissenschaftlicher Mitarbeiter ("`Forschungsgruppe Kappes"')}{\sout{wissenschaftlicher Mitarbeiter ("`Forschungsgruppe Kappes"')}}
        \item<5-> \temporal<5>{}{Masterstudium Intelligente Systeme (BaSys)}{\sout{Masterstudium Intelligente Systeme (BaSys)}}
        \item<7-> ab Oktober: Front-End Engineer bei Facebook, USA
    \end{itemize}
    
\end{frame}

\subsection{Versions-was?}

\begin{frame}{Git is ein Quelltextverwaltungs-System}
    \begin{itemize}[<+->]
        \item Archivierung, Verwaltung und Versionierung von Quelltexten
        \item Subversion, Team Foundation Server, Mercurial
        \item kolaboratives Arbeiten, Changelog, Rollbacks
    \end{itemize}
\end{frame}

\begin{frame}{Gits Herausstellugnsmerkmale}
    \begin{itemize}[<+->]
        \item \alert<6>{verteilt}
        \item klein \& schnell
        \item \alert<6>{sicher}
        \item \alert<6>{mächtig}
        \item frei
    \end{itemize}
\end{frame}

\begin{frame}{Wer benutzt Git bereits?}
    \centerline{\includegraphics[height=.8\textheight]{./git-users.png}}
\end{frame}

\section{Git Grundlagen}

\subsection{Verteilung}

\begin{frame}{Welche Vorteile bietet Verteilung?}
    \begin{itemize}[<+->]
        \item vollständige Kopie des Repository
        \item voller offline Zugriff (schnell!)
        \item direkter Austausch mit Peers
        \item natürlicher, Open Source freundlicher Workflow (web of trust)
    \end{itemize}
\end{frame}


\begin{frame}{Centralized Workflow (z.B. Subversion)}
    \centerline{\includegraphics[width=.8\textwidth]{./centralized-workflow.png}}
\end{frame}

\begin{frame}{Distributed Workflow (z.B. Git)}
    \centerline{\includegraphics[width=.8\textwidth]{./distributed-workflow.png}}
\end{frame}

\subsection{Integrität und Versionierung}

\begin{frame}{Rohre und Leitungen}
    \begin{itemize}[<+->]
        \item Snapshots
        \item Blobs, Trees, Commits
        \item Key-Value Store
        \item SHA1 für Integrität und als Bezeichner
        \item äußerst simpel, leicht zu implementieren
    \end{itemize}
\end{frame}

\begin{frame}{Dateien und Deltas (z.B. Subversion)}
    \centerline{\includegraphics[width=0.8\textwidth]{./deltas.png}}
\end{frame}

\begin{frame}{Blobs und Snapshots (z.B. Git)}
    \centerline{\includegraphics[width=.8\textwidth]{./snapshots.png}}
\end{frame}

\subsection{Staging Area}

\begin{frame}{Repository Modell}
    \begin{itemize}[<+->]
        \item 3-stufig
        \item working directory
        \item staging area (index)
        \item repository
        \item (remotes)
    \end{itemize}
\end{frame}

\begin{frame}{Bühne frei!}
    \centerline{\includegraphics[width=.8\textwidth]{./staging.png}}
\end{frame}

\begin{frame}{Bühne frei!}
    \centerline{\includegraphics[width=.8\textwidth]{./staging2.png}}
\end{frame}

\section{Git Praktisch}

\section{Schluss}
\subsection{}

\begin{frame}{Ressourcen}
    \begin{itemize}[<+->]
        \item \textcolor{blue}{Pro Git} Praktische Einführung in Git. \url{http://www.git-scm.com/book}
        \item \textcolor{blue}{Introduction to Git} Talk mit viel praktischem Know-How zu Git. \url{http://www.youtube.com/watch?v=ZDR433b0HJY}
        \item \textcolor{blue}{Google Tech Talk, Linus Tovalds} Talk mit (historischem) Hintergrund zu Git. \url{http://www.youtube.com/watch?v=4XpnKHJAok8}
        \item \textcolor{blue}{Git Reference} Git Referenz. \url{http://gitref.org/}
    \end{itemize}
\end{frame}

\begin{frame}{Tipps}
    \begin{itemize}[<+->]
        \item CLI/Kommandozeile
        \item Linux/Unix
        \item GitHub \url{http://www.github.com}
        \item Nutzt Git!
    \end{itemize}
\end{frame}

\end{document}
