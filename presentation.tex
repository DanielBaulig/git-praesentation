\newcommand{\command}[1]{\texttt{#1}}
\newcommand{\variable}[1]{\textit{#1}}

\begin{document}

\frame{
    \titlepage
}

\frame{
    \frametitle{Übersicht}
    \tableofcontents
}

\section{Einführung}
\subsection{Über mich}

\begin{frame}{Hello, my name is...}
    \begin{itemize}
        \item<1-> Daniel Baulig
        \item<2-> B.Sc. Informatik (TU Darmstadt, FH Frankfurt)
        \item<3-> IT-Security, Web (JS, HTML5, node.js), AI, FLOSS
        \item<4-> \temporal<4-5>{}{wissenschaftlicher Mitarbeiter ("`Forschungsgruppe Kappes"')}{\sout{wissenschaftlicher Mitarbeiter ("`Forschungsgruppe Kappes"')}}
        \item<5-> \temporal<5>{}{Masterstudium Intelligente Systeme (BaSys)}{\sout{Masterstudium Intelligente Systeme (BaSys)}}
        \item<7-> ab Oktober: Front-End Engineer bei Facebook, USA
    \end{itemize}

\end{frame}

\subsection{Versions-was?}

\begin{frame}{Git is ein Quelltextverwaltungs-System}
    \begin{itemize}[<+->]
        \item Archivierung, Verwaltung und Versionierung von Quelltexten
        \item Subversion, Team Foundation Server, Mercurial
        \item kolaboratives Arbeiten, Change Management, Attribution
    \end{itemize}
\end{frame}

\begin{frame}{Gits Herausstellugnsmerkmale}
    \begin{itemize}[<+->]
        \item \alert<6>{verteilt}
        \item klein \& schnell
        \item \alert<6>{sicher}
        \item \alert<6>{mächtig}
        \item frei
    \end{itemize}
\end{frame}

\begin{frame}{Wer benutzt Git bereits?}
    \centerline{\includegraphics[height=.8\textheight]{./git-users.png}}
\end{frame}

\section{Git Grundlagen}

\subsection{Verteilung}

\begin{frame}{Welche Vorteile bietet Verteilung?}
    \begin{itemize}[<+->]
        \item vollständige Kopie des Repository
        \item voller offline Zugriff (schnell!)
        \item direkter Austausch mit Peers
        \item natürlicher, Open Source freundlicher Workflow (web of trust)
    \end{itemize}
\end{frame}


\begin{frame}{Centralized Workflow (z.B. Subversion)}
    \centerline{\includegraphics[width=.8\textwidth]{./centralized-workflow.png}}
\end{frame}

\begin{frame}{Distributed Workflow (z.B. Git)}
    \centerline{\includegraphics[width=.8\textwidth]{./distributed-workflow.png}}
\end{frame}

\subsection{Integrität und Versionierung}

\begin{frame}{Rohre und Leitungen}
    \begin{itemize}[<+->]
        \item Snapshots
        \item Blobs, Trees, Commits
        \item Key-Value Store
        \item SHA1 für Integrität und als Bezeichner
        \item äußerst simpel, leicht zu implementieren
    \end{itemize}
\end{frame}

\begin{frame}{Dateien und Deltas (z.B. Subversion)}
    \centerline{\includegraphics[width=0.8\textwidth]{./deltas.png}}
\end{frame}

\begin{frame}{Blobs und Snapshots (z.B. Git)}
    \centerline{\includegraphics[width=.8\textwidth]{./snapshots.png}}
\end{frame}

\subsection{Staging Area}

\begin{frame}{Repository Modell}
    \begin{itemize}[<+->]
        \item 3-stufig
        \item working directory
        \item staging area (index)
        \item repository
        \item (remotes)
    \end{itemize}
\end{frame}

\begin{frame}{Bühne frei!}
    \centerline{\includegraphics[width=.8\textwidth]{./staging.png}}
\end{frame}

\begin{frame}{Bühne frei!}
    \centerline{\includegraphics[width=.8\textwidth]{./staging2.png}}
\end{frame}

\section{Git Praktisch}
\subsection{}

\begin{frame}{Setup}
    \begin{block}<1->{Identität festlegen}
\command{git config --global user.name "\variable{your name}"}\newline
\command{git config --global user.email "\variable{your@mail.com}"}
    \end{block}
    \vspace{2em}
    \begin{block}<2->{Lokales Repository anlegen}
        \command{git init}
    \end{block}
    \uncover<3->{\centerline{\textbf{ODER}}}
    \begin{block}<3->{Remote Repository klonen}
            \command{git clone \variable{remote-url}}
    \end{block}
\end{frame}

\begin{frame}{Änderungen hinzufügen}
    \begin{block}<1->{Änderungen vornehmen}
        \command{echo "Hello, World!"{} > \variable{file}}
    \end{block}
    \begin{block}<2->{Änderung stagen}
        \command{git add \variable{file}}
    \end{block}
    \begin{block}<3->{Gestagete Änderungen commiten}
        \command{git commit}
    \end{block}
\end{frame}

\begin{frame}{Branchen \& Mergen}
    \begin{block}<1->{Branch anlegen}
        \command{git branch \variable{topic}}\newline
        \command{git checkout \variable{topic}}
    \end{block}
    \begin{block}<2->{Änderung vornehmen}
        \command{echo "Hello, World!"{} >{}> \variable{file}}\newline
        \command{git commit -am "{}Add another line"}
    \end{block}
    \begin{block}<3->{Zurück mergen}
        \command{git checkout master}\newline
        \command{git merge \variable{topic}}\newline
        \command{git branch -d \variable{topic}}
    \end{block}
\end{frame}

\begin{frame}{Änderungen pushen\uncover<3->{ und fetchen}}
    \begin{block}{Commits pushen}
        \command{git push [\variable{remote-url}|\variable{remote-alias}] [\variable{branch}]}
    \end{block}
    \begin{alertblock}<2->{Kein Fast-Forward Merge?}
        \command{Failed to push some refs to <\variable{repo}>. To prevent you from losing history, non-fast-forward updates were rejected. Merge remote changes before pushing again.}
    \end{alertblock}

    \temporal<3>{\vspace{5em}}{\begin{block}{Remote Commits fetchen und mergen}
        \command{git fetch [\variable{remote-url}|\variable{remote-alias}]}\newline
        \command{git merge \{\variable{remote-alias}|\variable{remote-url}\}/\variable{branch}}
    \end{block}}{\begin{block}{Remote Commits pullen}
        \vspace{.6em}
        \command{git pull \{\variable{remote-alias}|\variable{remote-url}\} [\variable{branch}]}
        \vspace{.6em}
    \end{block}}

    \begin{block}<5->{Commits (erneut) pushen}
        \command{git push [\variable{remote-url}|\variable{remote-alias}] [\variable{branch}]}
    \end{block}
\end{frame}

\begin{frame}{Git kann (viel) mehr!}
    \begin{columns}
        \begin{column}{.8\textwidth}
        \begin{itemize}[<+->]
            \item[git status] Informationen zum Working-Directory
            \item[git log] Zeigt die Commit-History
            \item[git diff] Vergleicht zwei Objekte
            \item[git rebase] Replay-Merge + History Rewriting
            \item[git stash] Working-Directory Stack
            \item[git reset] Working-Directory und Index Manipulation
            \item[git help] Sehr gute Online-Hilfe
        \end{itemize}
        \end{column}
    \end{columns}
\end{frame}

\section{Schluss}
\subsection{What next?}

\begin{frame}{Ressourcen}
    \begin{itemize}[<+->]
        \item \textcolor{blue}{Pro Git} Praktische Einführung in Git. \url{http://www.git-scm.com/book}
        \item \textcolor{blue}{Introduction to Git} Talk mit viel praktischem Know-How zu Git. \url{http://www.youtube.com/watch?v=ZDR433b0HJY}
        \item \textcolor{blue}{Google Tech Talk, Linus Tovalds} Talk mit (historischem) Hintergrund zu Git. \url{http://www.youtube.com/watch?v=4XpnKHJAok8}
        \item \textcolor{blue}{Git Reference} Git Referenz. \url{http://gitref.org/}
    \end{itemize}
\end{frame}

\begin{frame}{Tipps}
    \begin{itemize}[<+->]
        \item CLI/Kommandozeile
        \item Linux/Unix
        \item GitHub \url{http://www.github.com}
        \item \textbf{Nutzt Git!}
    \end{itemize}
\end{frame}

\subsection{The End}

\begin{frame}{Kontakt}
    \begin{columns}
        \begin{column}{.8\textwidth}
            \begin{itemize}
                \item [Twitter] @danielbaulig
                \item [Facebook] facebook.com/dbaulig
                \item [Mail] daniel.baulig@gmx.de
                \item [GitHub] github.com/DanielBaulig
            \end{itemize}
        \end{column}
    \end{columns}
\end{frame}

\begin{frame}{Danke!}
    \centerline{\Huge{Danke!}}
\end{frame}

\end{document}
