\begin{document}

\frame{
    \titlepage
}

\section{Einführung}
\subsection{}

\begin{frame}{Git is ein Quelltextverwaltungs-System}
    \begin{itemize}[<+->]
        \item Archivierung, Verwaltung und Versionierung von Quelltexten
        \item Subversion, Team Foundation Server, Mercurial
        \item kolaboratives Arbeiten, Changelog, Rollbacks
    \end{itemize}
\end{frame}

\begin{frame}{Gits Herausstellugnsmerkmale}
    \begin{itemize}[<+->]
        \item \alert<6>{verteilt}
        \item klein \& schnell
        \item \alert<6>{sicher}
        \item \alert<6>{mächtig}
        \item frei
    \end{itemize}
\end{frame}

\begin{frame}{Wer benutzt Git bereits?}
    \centerline{\includegraphics[height=.8\textheight]{./git-users.png}}
\end{frame}

\section{Git Grundlagen}

\subsection{Verteilung}

\begin{frame}{Welche Vorteile bietet Verteilung?}
    \begin{itemize}[<+->]
        \item vollständige Kopie des Repository
        \item voller offline Zugriff (schnell!)
        \item direkter Austausch mit Peers
        \item natürlicher, Open Source freundlicher Workflow (web of trust)
    \end{itemize}
\end{frame}


\begin{frame}{Centralized Workflow (z.B. Subversion)}
    \centerline{\includegraphics[width=.8\textwidth]{./centralized-workflow.png}}
\end{frame}

\begin{frame}{Distributed Workflow (z.B. Git)}
    \centerline{\includegraphics[width=.8\textwidth]{./distributed-workflow.png}}
\end{frame}

\subsection{Integrität und Versionierung}

\begin{frame}{Rohre und Leitungen}
    \begin{itemize}[<+->]
        \item Snapshots
        \item Blobs, Trees, Commits
        \item Key-Value Store
        \item SHA1 für Integrität und als Bezeichner
        \item äußerst simpel, leicht zu implementieren
    \end{itemize}
\end{frame}

\begin{frame}{Dateien und Deltas (z.B. Subversion)}
    \centerline{\includegraphics[width=0.8\textwidth]{./deltas.png}}
\end{frame}

\begin{frame}{Blobs und Snapshots (z.B. Git)}
    \centerline{\includegraphics[width=.8\textwidth]{./snapshots.png}}
\end{frame}

\end{document}
